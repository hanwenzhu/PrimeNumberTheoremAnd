
\inputleannode{finsum_range_eq_sum_range}





\inputleannode{chebyshev_asymptotic}





\inputleannode{primorial_bounds}





\inputleannode{pi_asymp}





\inputleannode{pi_alt}





Let $p_n$ denote the $n^{th}$ prime.

\inputleannode{pn_asymptotic}





\inputleannode{pn_pn_plus_one}





\inputleannode{prime_between}





\inputleannode{mun}





\inputleannode{mu-pnt}





\inputleannode{lambda-pnt}






\inputleannode{mu-pnt-alt}





\section{Consequences of the PNT in arithmetic progressions}

\begin{theorem}[Prime number theorem in AP]\label{chebyshev_asymptotic_pnt}\lean{chebyshev_asymptotic_pnt}\leanok  If $a\ (q)$ is a primitive residue class, then one has
  $$ \sum_{p \leq x: p = a\ (q)} \log p = \frac{x}{\phi(q)} + o(x).$$
\end{theorem}


\begin{proof}
\uses{chebyshev_asymptotic}
This is a routine modification of the proof of Theorem \ref{chebyshev_asymptotic}.
\end{proof}


\begin{corollary}[Dirichlet's theorem]\label{dirichlet_thm}\lean{dirichlet_thm}\leanok  Any primitive residue class contains an infinite number of primes.
\end{corollary}


\begin{proof}
\uses{chebyshev_asymptotic_pnt}
If this were not the case, then the sum $\sum_{p \leq x: p = a\ (q)} \log p$ would be bounded in $x$, contradicting Theorem \ref{chebyshev_asymptotic_pnt}.
\end{proof}
-/

/-%%
\section{Consequences of the Chebotarev density theorem}



\begin{lemma}[Cyclotomic Chebotarev]\label{Chebotarev-cyclic}  For any $a$ coprime to $m$,
$$ \sum_{N \mathfrak{p} \leq x; N \mathfrak{p} = a\ (m)} \log N \mathfrak{p}  =
\frac{1}{|G|} \sum_{N \mathfrak{p} \leq x} \log N \mathfrak{p}.$$
\end{lemma}


\begin{proof}\uses{Dedekind-PNT, WeakPNT_AP} This should follow from Lemma \ref{Dedekind-PNT} by a Fourier expansion.
\end{proof}

