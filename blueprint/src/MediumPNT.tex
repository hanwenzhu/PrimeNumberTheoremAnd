
The approach here is completely standard. We follow the use of
$\mathcal{M}(\widetilde{1_{\epsilon}})$ as in [Kontorovich 2015].


\begin{definition}\label{ChebyshevPsi}\lean{ChebyshevPsi}\leanok
The (second) Chebyshev Psi function is defined as
$$
\psi(x) := \sum_{n \le x} \Lambda(n),
$$
where $\Lambda(n)$ is the von Mangoldt function.
\end{definition}


It has already been established that zeta doesn't vanish on the 1 line, and has a pole at $s=1$
of order 1.
We also have the following.
\begin{theorem}[LogDerivativeDirichlet]\label{LogDerivativeDirichlet}\lean{LogDerivativeDirichlet}\leanok
We have that, for $\Re(s)>1$,
$$
-\frac{\zeta'(s)}{\zeta(s)} = \sum_{n=1}^\infty \frac{\Lambda(n)}{n^s}.
$$
\end{theorem}


\begin{proof}\leanok
Already in Mathlib.
\end{proof}


The main object of study is the following inverse Mellin-type transform, which will turn out to
be a smoothed Chebyshev function.

\inputleannode{SmoothedChebyshev}


\begin{lemma}[SmoothedChebyshevIntegrand_conj]\label{SmoothedChebyshevIntegrand_conj}\lean{SmoothedChebyshevIntegrand_conj}\leanok
The smoothed Chebyshev integrand satisfies the conjugation symmetry
$$
\psi_{\epsilon}(X)(\overline{s}) = \overline{\psi_{\epsilon}(X)(s)}
$$
for all $s \in \mathbb{C}$, $X > 0$, and $\epsilon > 0$.
\end{lemma}


\begin{proof}\leanok
\ uses{deriv_riemannZeta_conj, riemannZeta_conj}
We expand the definition of the smoothed Chebyshev integrand and compute, using the corresponding
conjugation symmetries of the Riemann zeta function and its derivative.
\end{proof}


\inputleannode{SmoothedChebyshevDirichlet_aux_integrable}





\inputleannode{SmoothedChebyshevDirichlet_aux_tsum_integral}





Inserting the Dirichlet series expansion of the log derivative of zeta, we get the following.
\inputleannode{SmoothedChebyshevDirichlet}





The smoothed Chebyshev function is close to the actual Chebyshev function.
\inputleannode{SmoothedChebyshevClose}





Returning to the definition of $\psi_{\epsilon}$, fix a large $T$ to be chosen later, and set
$\sigma_0 = 1 + 1 / log X$,
$\sigma_1 = 1- A/ \log T^9$, and
$\sigma_2<\sigma_1$ a constant.
Pull
contours (via rectangles!) to go
from $\sigma_0-i\infty$ up to $\sigma_0-iT$, then over to $\sigma_1-iT$,
up to $\sigma_1-3i$, over to $\sigma_2-3i$, up to $\sigma_2+3i$, back over to $\sigma_1+3i$, up to $\sigma_1+iT$, over to $\sigma_0+iT$, and finally up to $\sigma_0+i\infty$.

\begin{verbatim}
                    |
                    | I₉
              +-----+
              |  I₈
              |
           I₇ |
              |
              |
  +-----------+
  |       I₆
I₅|
--σ₂----------σ₁--1-σ₀----
  |
  |       I₄
  +-----------+
              |
              |
            I₃|
              |
              |  I₂
              +-----+
                    | I₁
                    |
\end{verbatim}

In the process, we will pick up the residue at $s=1$.
We will do this in several stages. Here the interval integrals are defined as follows:


\inputleannode{I1}


\inputleannode{I2}


\inputleannode{I37}


\inputleannode{I8}


\inputleannode{I9}


\inputleannode{I3}

\inputleannode{I7}


\inputleannode{I4}


\inputleannode{I6}


\inputleannode{I5}


\inputleannode{dlog_riemannZeta_bdd_on_vertical_lines}





\inputleannode{SmoothedChebyshevPull1_aux_integrable}





\inputleannode{BddAboveOnRect}





\inputleannode{SmoothedChebyshevPull1}





Next pull contours to another box.
\inputleannode{SmoothedChebyshevPull2}





We insert this information in $\psi_{\epsilon}$. We add and subtract the integral over the box
$[1-\delta,2] \times_{ℂ} [-T,T]$, which we evaluate as follows
\inputleannode{ZetaBoxEval}





It remains to estimate all of the integrals.


This auxiliary lemma is useful for what follows.
\inputleannode{IBound_aux1}





\inputleannode{I1Bound}





\inputleannode{I2Bound}





\inputleannode{I8I2}





\inputleannode{I8Bound}





\inputleannode{IntegralofLogx^n/x^2Bounded}





\inputleannode{I3Bound}





\inputleannode{I4Bound}





\inputleannode{I5Bound}





\section{MediumPNT}

\inputleannode{MediumPNT}




