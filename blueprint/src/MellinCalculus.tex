
\inputleannode{PartialIntegration}





In this section, we define the Mellin transform (already in Mathlib, thanks to David Loeffler),
prove its inversion formula, and
derive a number of important properties of some special functions and bumpfunctions.

Def: (Already in Mathlib)
Let $f$ be a function from $\mathbb{R}_{>0}$ to $\mathbb{C}$. We define the Mellin transform of
$f$ to be the function $\mathcal{M}(f)$ from $\mathbb{C}$ to $\mathbb{C}$ defined by
$$\mathcal{M}(f)(s) = \int_0^\infty f(x)x^{s-1}dx.$$

[Note: My preferred way to think about this is that we are integrating over the multiplicative
group $\mathbb{R}_{>0}$, multiplying by a (not necessarily unitary!) character $|\cdot|^s$, and
integrating with respect to the invariant Haar measure $dx/x$. This is very useful in the kinds
of calculations carried out below. But may be more difficult to formalize as things now stand. So
we might have clunkier calculations, which ``magically'' turn out just right - of course they're
explained by the aforementioned structure...]



Finally, we need Mellin Convolutions and properties thereof.
\inputleannode{MellinConvolution}


Let us start with a simple property of the Mellin convolution.
\inputleannode{MellinConvolutionSymmetric}





The Mellin transform of a convolution is the product of the Mellin transforms.
\inputleannode{MellinConvolutionTransform}





The $\nu$ function has Mellin transform $\mathcal{M}(\nu)(s)$ which is entire and decays (at
least) like $1/|s|$.
\inputleannode{MellinOfPsi}

[Of course it decays faster than any power of $|s|$, but it turns out that we will just need one
power.]





We can make a delta spike out of this bumpfunction, as follows.
\inputleannode{DeltaSpike}


This spike still has mass one:
\inputleannode{DeltaSpikeMass}





The Mellin transform of the delta spike is easy to compute.
\inputleannode{MellinOfDeltaSpike}





In particular, for $s=1$, we have that the Mellin transform of $\nu_\epsilon$ is $1+O(\epsilon)$.
\inputleannode{MellinOfDeltaSpikeAt1}





\inputleannode{MellinOfDeltaSpikeAt1_asymp}





Let $1_{(0,1]}$ be the function from $\mathbb{R}_{>0}$ to $\mathbb{C}$ defined by
$$1_{(0,1]}(x) = \begin{cases}
1 & \text{ if }x\leq 1\\
0 & \text{ if }x>1
\end{cases}.$$
This has Mellin transform
\inputleannode{MellinOf1}
[Note: this already exists in mathlib]





What will be essential for us is properties of the smooth version of $1_{(0,1]}$, obtained as the
 Mellin convolution of $1_{(0,1]}$ with $\nu_\epsilon$.
\inputleannode{Smooth1}





In particular, we have the following two properties.
\inputleannode{Smooth1Properties_below}





\inputleannode{Smooth1Properties_above}





\inputleannode{Smooth1Nonneg}





\inputleannode{Smooth1LeOne}





Combining the above, we have the following three Main Lemmata of this section on the Mellin
transform of $\widetilde{1_{\epsilon}}$.
\inputleannode{MellinOfSmooth1a}





\inputleannode{MellinOfSmooth1b}





\inputleannode{MellinOfSmooth1c}





\inputleannode{Smooth1ContinuousAt}




