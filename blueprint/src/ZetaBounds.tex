
We record here some prelimiaries about the zeta function and general
holomorphic functions.

\inputleannode{ResidueOfTendsTo}





\inputleannode{riemannZetaResidue}





\inputleannode{nonZeroOfBddAbove}


  
  

\inputleannode{logDerivResidue}





\inputleannode{BddAbove_to_IsBigO}





Let's also record that if a function $f$ has a simple pole at $p$ with residue $A$, and $g$ is holomorphic near $p$, then the residue of $f \cdot g$ is $A \cdot g(p)$.
\inputleannode{ResidueMult}





As a corollary, the log derivative of the Riemann zeta function has a simple pole at $s=1$:
\inputleannode{riemannZetaLogDerivResidue}





\inputleannode{riemannZeta0}


\inputleannode{sum_eq_int_deriv}





\inputleannode{ZetaSum_aux1}





\inputleannode{ZetaBnd_aux1a}





\inputleannode{ZetaSum_aux2}





\inputleannode{ZetaBnd_aux1b}





\inputleannode{ZetaBnd_aux1}





Big-Oh version of Lemma \ref{ZetaBnd_aux1}.
\inputleannode{ZetaBnd_aux1p}





\begin{lemma}[HolomorphicOn_Zeta0]\label{HolomorphicOn_Zeta0}\lean{HolomorphicOn_Zeta0}\leanok
For any $N\ge1$, the function $\zeta_0(N,s)$ is holomorphic on $\{s\in \C\mid \Re(s)>0 ∧ s \ne 1\}$.
\end{lemma}


\begin{proof}\uses{riemannZeta0, ZetaBnd_aux1b}\leanok
  The function $\zeta_0(N,s)$ is a finite sum of entire functions, plus an integral
  that's absolutely convergent on $\{s\in \C\mid \Re(s)>0 ∧ s \ne 1\}$ by Lemma \ref{ZetaBnd_aux1b}.
\end{proof}


\inputleannode{isPathConnected_aux}





\inputleannode{Zeta0EqZeta}





\inputleannode{ZetaBnd_aux2}





\inputleannode{ZetaUpperBnd}





\inputleannode{DerivUpperBnd_aux7}





\inputleannode{ZetaDerivUpperBnd}





\inputleannode{ZetaNear1BndFilter}





\inputleannode{ZetaNear1BndExact}





\inputleannode{ZetaLowerBound3}





\inputleannode{ZetaInvBound1}





\inputleannode{ZetaInvBound2}





\inputleannode{Zeta_eq_int_derivZeta}





\inputleannode{Zeta_diff_Bnd}





\inputleannode{ZetaInvBnd}





Annoyingly, it is not immediate from this that $\zeta$ doesn't vanish there! That's because
$1/0 = 0$ in Lean. So we give a second proof of the same fact (refactor this later), with a lower
 bound on $\zeta$ instead of upper bound on $1 / \zeta$.
\inputleannode{ZetaLowerBnd}





Now we get a zero free region.
\inputleannode{ZetaZeroFree}





\inputleannode{LogDerivZetaBnd}





\inputleannode{ZetaNoZerosOn1Line}





Then, since $\zeta$ doesn't vanish on the 1-line, there is a $\sigma<1$ (depending on $T$), so that
the box $[\sigma,1] \times_{ℂ} [-T,T]$ is free of zeros of $\zeta$.
\inputleannode{ZetaNoZerosInBox}





We now prove that there's an absolute constant $\sigma_0$ so that $\zeta'/\zeta$ is holomorphic on a rectangle $[\sigma_2,2] \times_{ℂ} [-3,3] \setminus \{1\}$.
\inputleannode{LogDerivZetaHolcSmallT}





\inputleannode{LogDerivZetaHolcLargeT}





\inputleannode{LogDerivZetaBndUnif}


\begin{proof}\uses{LogDerivZetaBnd}\leanok
For $\sigma$ close to $1$ use Lemma \ref{LogDerivZetaBnd}, otherwise estimate trivially.

